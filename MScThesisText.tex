% Options for packages loaded elsewhere
\PassOptionsToPackage{unicode}{hyperref}
\PassOptionsToPackage{hyphens}{url}
%
\documentclass[
  man,floatsintext]{apa7}
\usepackage{amsmath,amssymb}
\usepackage{lmodern}
\usepackage{iftex}
\ifPDFTeX
  \usepackage[T1]{fontenc}
  \usepackage[utf8]{inputenc}
  \usepackage{textcomp} % provide euro and other symbols
\else % if luatex or xetex
  \usepackage{unicode-math}
  \defaultfontfeatures{Scale=MatchLowercase}
  \defaultfontfeatures[\rmfamily]{Ligatures=TeX,Scale=1}
\fi
% Use upquote if available, for straight quotes in verbatim environments
\IfFileExists{upquote.sty}{\usepackage{upquote}}{}
\IfFileExists{microtype.sty}{% use microtype if available
  \usepackage[]{microtype}
  \UseMicrotypeSet[protrusion]{basicmath} % disable protrusion for tt fonts
}{}
\makeatletter
\@ifundefined{KOMAClassName}{% if non-KOMA class
  \IfFileExists{parskip.sty}{%
    \usepackage{parskip}
  }{% else
    \setlength{\parindent}{0pt}
    \setlength{\parskip}{6pt plus 2pt minus 1pt}}
}{% if KOMA class
  \KOMAoptions{parskip=half}}
\makeatother
\usepackage{xcolor}
\IfFileExists{xurl.sty}{\usepackage{xurl}}{} % add URL line breaks if available
\IfFileExists{bookmark.sty}{\usepackage{bookmark}}{\usepackage{hyperref}}
\hypersetup{
  pdftitle={Newly Left-Behind Children's Educational Performance Drops Slightly After Parents Migrated},
  pdfauthor={Jiancheng Gu1},
  pdflang={en-EN},
  pdfkeywords={keywords},
  hidelinks,
  pdfcreator={LaTeX via pandoc}}
\urlstyle{same} % disable monospaced font for URLs
\usepackage{longtable,booktabs,array}
\usepackage{calc} % for calculating minipage widths
% Correct order of tables after \paragraph or \subparagraph
\usepackage{etoolbox}
\makeatletter
\patchcmd\longtable{\par}{\if@noskipsec\mbox{}\fi\par}{}{}
\makeatother
% Allow footnotes in longtable head/foot
\IfFileExists{footnotehyper.sty}{\usepackage{footnotehyper}}{\usepackage{footnote}}
\makesavenoteenv{longtable}
\usepackage{graphicx}
\makeatletter
\def\maxwidth{\ifdim\Gin@nat@width>\linewidth\linewidth\else\Gin@nat@width\fi}
\def\maxheight{\ifdim\Gin@nat@height>\textheight\textheight\else\Gin@nat@height\fi}
\makeatother
% Scale images if necessary, so that they will not overflow the page
% margins by default, and it is still possible to overwrite the defaults
% using explicit options in \includegraphics[width, height, ...]{}
\setkeys{Gin}{width=\maxwidth,height=\maxheight,keepaspectratio}
% Set default figure placement to htbp
\makeatletter
\def\fps@figure{htbp}
\makeatother
\setlength{\emergencystretch}{3em} % prevent overfull lines
\providecommand{\tightlist}{%
  \setlength{\itemsep}{0pt}\setlength{\parskip}{0pt}}
\setcounter{secnumdepth}{-\maxdimen} % remove section numbering
% Make \paragraph and \subparagraph free-standing
\ifx\paragraph\undefined\else
  \let\oldparagraph\paragraph
  \renewcommand{\paragraph}[1]{\oldparagraph{#1}\mbox{}}
\fi
\ifx\subparagraph\undefined\else
  \let\oldsubparagraph\subparagraph
  \renewcommand{\subparagraph}[1]{\oldsubparagraph{#1}\mbox{}}
\fi
\newlength{\cslhangindent}
\setlength{\cslhangindent}{1.5em}
\newlength{\csllabelwidth}
\setlength{\csllabelwidth}{3em}
\newlength{\cslentryspacingunit} % times entry-spacing
\setlength{\cslentryspacingunit}{\parskip}
\newenvironment{CSLReferences}[2] % #1 hanging-ident, #2 entry spacing
 {% don't indent paragraphs
  \setlength{\parindent}{0pt}
  % turn on hanging indent if param 1 is 1
  \ifodd #1
  \let\oldpar\par
  \def\par{\hangindent=\cslhangindent\oldpar}
  \fi
  % set entry spacing
  \setlength{\parskip}{#2\cslentryspacingunit}
 }%
 {}
\usepackage{calc}
\newcommand{\CSLBlock}[1]{#1\hfill\break}
\newcommand{\CSLLeftMargin}[1]{\parbox[t]{\csllabelwidth}{#1}}
\newcommand{\CSLRightInline}[1]{\parbox[t]{\linewidth - \csllabelwidth}{#1}\break}
\newcommand{\CSLIndent}[1]{\hspace{\cslhangindent}#1}
\ifLuaTeX
\usepackage[bidi=basic]{babel}
\else
\usepackage[bidi=default]{babel}
\fi
\babelprovide[main,import]{english}
% get rid of language-specific shorthands (see #6817):
\let\LanguageShortHands\languageshorthands
\def\languageshorthands#1{}
% Manuscript styling
\usepackage{upgreek}
\captionsetup{font=singlespacing,justification=justified}

% Table formatting
\usepackage{longtable}
\usepackage{lscape}
% \usepackage[counterclockwise]{rotating}   % Landscape page setup for large tables
\usepackage{multirow}		% Table styling
\usepackage{tabularx}		% Control Column width
\usepackage[flushleft]{threeparttable}	% Allows for three part tables with a specified notes section
\usepackage{threeparttablex}            % Lets threeparttable work with longtable

% Create new environments so endfloat can handle them
% \newenvironment{ltable}
%   {\begin{landscape}\centering\begin{threeparttable}}
%   {\end{threeparttable}\end{landscape}}
\newenvironment{lltable}{\begin{landscape}\centering\begin{ThreePartTable}}{\end{ThreePartTable}\end{landscape}}

% Enables adjusting longtable caption width to table width
% Solution found at http://golatex.de/longtable-mit-caption-so-breit-wie-die-tabelle-t15767.html
\makeatletter
\newcommand\LastLTentrywidth{1em}
\newlength\longtablewidth
\setlength{\longtablewidth}{1in}
\newcommand{\getlongtablewidth}{\begingroup \ifcsname LT@\roman{LT@tables}\endcsname \global\longtablewidth=0pt \renewcommand{\LT@entry}[2]{\global\advance\longtablewidth by ##2\relax\gdef\LastLTentrywidth{##2}}\@nameuse{LT@\roman{LT@tables}} \fi \endgroup}

% \setlength{\parindent}{0.5in}
% \setlength{\parskip}{0pt plus 0pt minus 0pt}

% Overwrite redefinition of paragraph and subparagraph by the default LaTeX template
% See https://github.com/crsh/papaja/issues/292
\makeatletter
\renewcommand{\paragraph}{\@startsection{paragraph}{4}{\parindent}%
  {0\baselineskip \@plus 0.2ex \@minus 0.2ex}%
  {-1em}%
  {\normalfont\normalsize\bfseries\itshape\typesectitle}}

\renewcommand{\subparagraph}[1]{\@startsection{subparagraph}{5}{1em}%
  {0\baselineskip \@plus 0.2ex \@minus 0.2ex}%
  {-\z@\relax}%
  {\normalfont\normalsize\itshape\hspace{\parindent}{#1}\textit{\addperi}}{\relax}}
\makeatother

% \usepackage{etoolbox}
\makeatletter
\patchcmd{\HyOrg@maketitle}
  {\section{\normalfont\normalsize\abstractname}}
  {\section*{\normalfont\normalsize\abstractname}}
  {}{\typeout{Failed to patch abstract.}}
\patchcmd{\HyOrg@maketitle}
  {\section{\protect\normalfont{\@title}}}
  {\section*{\protect\normalfont{\@title}}}
  {}{\typeout{Failed to patch title.}}
\makeatother

\usepackage{xpatch}
\makeatletter
\xapptocmd\appendix
  {\xapptocmd\section
    {\addcontentsline{toc}{section}{\appendixname\ifoneappendix\else~\theappendix\fi\\: #1}}
    {}{\InnerPatchFailed}%
  }
{}{\PatchFailed}
\keywords{keywords\newline\indent Word count: X}
\usepackage{lineno}

\linenumbers
\usepackage{csquotes}
\makeatletter
\renewcommand{\paragraph}{\@startsection{paragraph}{4}{\parindent}%
  {0\baselineskip \@plus 0.2ex \@minus 0.2ex}%
  {-1em}%
  {\normalfont\normalsize\bfseries\typesectitle}}

\renewcommand{\subparagraph}[1]{\@startsection{subparagraph}{5}{1em}%
  {0\baselineskip \@plus 0.2ex \@minus 0.2ex}%
  {-\z@\relax}%
  {\normalfont\normalsize\bfseries\itshape\hspace{\parindent}{#1}\textit{\addperi}}{\relax}}
\makeatother

\ifLuaTeX
  \usepackage{selnolig}  % disable illegal ligatures
\fi

\title{Newly Left-Behind Children's Educational Performance Drops Slightly After Parents Migrated}
\author{Jiancheng Gu\textsuperscript{1}}
\date{}


\shorttitle{Left-Behind Children}

\affiliation{\vspace{0.5cm}\textsuperscript{1} Faculty of Social Sciences, Vrije Universiteit Amsterdam}

\abstract{%
Millions of children worldwide are ``left behind'' when their parents migrate away for work.
Parents' labor migration increases household income but decreases parental care, thereby exerting mixed influences on child development, including educational performance.
To what extent do different arrangements of parental migration affect children's schooling performance in the short term?
Studying the case of China, I obtained a sample of four thousand junior high school students from a nationally representative two-wave panel survey.
I differentiated household migration arrangements of any one parent, of mother only, of father only, and of both parents.
I measured children's educational performance by standardized cognitive test score and academic exam grades.
For causal inference, I built difference-in-differences models with propensity score matching.
Results show that the new left-behind children's performance at school dropped slightly compared to that of non-left-behind peers.
Father-only migration is negatively associated with children's cognitive abilities, while mother-only migration is negatively associated with children's academic grades.
}



\begin{document}
\maketitle

\hypertarget{introduction}{%
\section{Introduction}\label{introduction}}

Millions of children are resided in their home communities but separated from their parents as either one or both parents have migrated for work. These are the so-called ``left-behind children'' or ``stay-behind children.'' They are common in migrant-sending regions, such as Latin America, Sub-Sahara Africa, East Europe, and large parts of Asia. Researchers estimated that one-sixth of Mexican children (DeWaard et al., 2018), one-fifth of Bulgarian children (Popova, 2018), and one-fourth of Chinese children were living apart from their migrating parents (NBS et al., 2017).

How are stay-behind children faring when parents have migrated? Scholars have obtained mixed results due to the differential effects of parental migration. Consider children's education. According to the review articles by Brauw (2019) and Van Hook and Glick (2020), labor migrants send remittances back, which can fund children's schooling as well as improve their nutritional and living conditions. The inflow of remittances, however, cannot substitute for the loss of parental care. Parental migration decreases the amount and quality of parental guidance, protection, and support. It is uncertain how these two opposing effects together shape children's educational performance, which merits empirical investigation.

A challenge arises when we want to assess the causality between parent's migration and children's educational achievement. Experiments can unveil causal relations, but it is unfeasible and unethical to design an experiment that randomly assigns parents to migrate without their children. Alternatively, we can infer causality from a quasi-experimental design -- that is, when the cause is on par with random assignment to participants in the analysis, conditional on identification assumptions (Cunningham, 2021).

In this thesis, my research question is: To what extent do parental migration affect the educational performance of left-behind children in the short term? I chose China as the case and analyzed data from a two-wave panel survey. The nationally representative sample includes children in both urban and rural areas. I developed a difference-in-differences model for causal inference and applied propensity score matching for robustness check. The outcome was measured with children's cognitive test score and academic exam scores. I demonstrate that parental migration reduces the educational performance of children staying behind, but only to a small extent.

\newpage

\hypertarget{literature-review}{%
\section{Literature Review}\label{literature-review}}

\hypertarget{theoretical-framework}{%
\subsection{Theoretical Framework}\label{theoretical-framework}}

Between parental migration and children's education, the mechanisms trace to two theoretical frameworks. One is the resource generation model. It holds that parental migration brings forth greater resources in the form of economic capital. Economic capital is, in Bourdieu's words, ``immediately and directly convertible into money and may be institutionalized in the form of property rights'' (2002, p. 16). Economic capital is transferable to human capital, which is the stock of ``skills, talents, health, and expertise'' in people that enhance their productivity (Botev et al., 2019; Goldin, 2016). D. Yang (2008) noted that in the Philippes, a sharp depreciation of peso increased the amount of remittances from migrant laborers working abroad. This subsequently boosted children's school enrollment and educational expenditure. In China, Chang et al. (2019) interviewed migrant workers and their family members. Oftentimes, they found that labor migration presented the only way to sustain their children's education and livelihood.

The other theoretical model is the family disruption model. Children flourish under parental nurturing care. This term refers to the ``stable environments that promote health and adequate nutrition, protect from threats, and provide opportunities for learning and responsive, emotionally supportive and developmentally enriching relationships'' (Black et al., 2021, p. 1). Parental migration decreases nurturing care, thereby hindering children's developmental outcomes. The loss of nurturing care slows children's accumulation of human capital. Hong talked to a group of left-behind children at a rural school (Hong \& Fuller, 2019). These students expressed the utter sense of loneliness ``with no advocate, no support and with no one looking out for them'' (p.~13). Many of them felt indifferent to school grades, and some even longed to ditch school to pursue a free, independent life (p.~14).

Researchers obtained mixed results regarding how parental migration affects children's education. Botezat and Pfeiffer (2020) studied the case of Romania and found a positive effect of parents' migration on academic grades of adolescents aged eleven to fifteen. L. Wang et al. (2019) collected data from junior high school students in central China and found no effect of parental on left-behind children's achievement in mathematical test. Studying four Chinese provinces, Brauw and Giles (2017) showed that as ample migration opportunities appeared in villages, enrollment to senior high school decreased. Nguyen (2016) demonstrated that parental migration was associated with lower cognitive capabilities among Indian and Vietnamese children aged five to eight. The inconsistency can result from factors such as the left-behind children's age and gender, the migrant parent's original domicile and eventual destination, and the arrangement of which parent migrate for how long.

Parental migration immediately disrupts family structure, which in turn lowers academic performance of children. It takes longer time for parental migration to generate resources that improve children's educational attainment. I propose the first hypothesis:

\(H_1\) In the short term, parental migration negatively affect the cognitive development and academic performance of children left-behind.

Researchers have studied whether the impact differ between mother absence and father absence in the context of labor migration. Y. Xu et al. (2019) adopted a fixed-effects propensity score weighting model on cross-sectional data collected from seventh and ninth graders in rural areas across China. They found that the absence of father only or both parents had ``little or no association with negative outcomes'' on children's academic and cognitive performances. Only-mother absence, however, showed a ``strong association with negative outcomes'' (p.~1646). Chen et al. (2019) conducted structural equation modeling on the cross-sectional data collected from fourth to seventh graders in the countryside of one province. They found that mother migration was ``negatively associated with children's social competence and academic performance,'' while father migration had no direct effects on the two outcomes (pp.~860-861).

In China, the social norms designate men as breadwinners and women as caregivers. Studies noted that in some Asian and Latin American societies, children are more likely to accept fathers' migration than mothers' (Murphy, 2022). Left-behind children may even resent their migrant mothers, viewing them as transgressing on the expected care-giving roles. I propose the second hypothesis:

\(H_2\) In the short term, mother migration negatively affect the cognitive development and academic performance of children left-behind, while father migration does not show a effect.

\hypertarget{chinas-context}{%
\subsection{China's Context}\label{chinas-context}}

Three features make China a suitable case for studying left-behind children. The first feature is the huge scale of internal labor migration. According to the National Bureau of Statistics, the year 2021 recorded 292.5 million in-country migrant workers (NBS, 2022). They form the backbone of essential industries, such as retail, delivery, transportation, and manufacturing. The number of left-behind Chinese children reached 68.7 million in 2015 (NBS et al., 2017). The second feature is the neoliberal welfare regime. Neoliberalism extends the logic of the market to most spheres of human life. Driven by the neoliberal doctrine, the government shifts its responsibilities of social services and welfare to individuals and communities. Since China initiated pro-market reforms in the 1980s, the state has reduced its welfare provisions and passed them to individuals and families (Duckett, 2020). The third is the authoritarian control against citizens. According to Gu (2021) and Zhang (2018), the hukou (household registration) system controls people's mobility by binding a household's public welfare access with its place of registry. One can move out of town freely but cannot change hukou freely. For rural household members moving to the city, it takes months -- if not years -- to apply for a new hukou. Before receiving urban hukou, they only enjoy limited access to public service in the host city, including public education for children. Migrant laborers had organized strikes and protests, only to meet with police crackdown (Chan \& Selden, 2017). Facing financial and policy constraints, many migrant laborers have to leave children in their hometowns (Gu, 2021; Zhang, 2018).

The existing research about China's left-behind children has three major shortcomings. The first is that most literature focuses on children left behind in rural areas while neglecting the urban counterparts. The National Bureau of Statistics stated that the number of urban left-behind children had been growing and expected that the trend would continue (NBS et al., 2017). My sample includes urban left-behind children as well. The second shortcoming lies in sampling and research design. Some studies suffer from a limited sample size of less than one thousand participants or a limited geographic coverage of one to a few towns (Jin et al., 2020; Y. Liu et al., 2021; Shu, 2021; F. Yang et al., 2022). Others only estimate correlation using cross-sectional data (Akezhuoli et al., 2022; Jin et al., 2020). These studies could have applied causal inference techniques like fixed effect, matching, or inverse probability of treatment weighting. Others s The third shortcoming lies in the measurement of outcome. Chang et al. (2019) and L. Wang et al. (2019) measured students' educational performance with the test score of only one academic subject, namely mathematics. To tackle these limitations, I adopted a nationally representative two-wave panel dataset. I included the test results of multiple subjects to measure comprehensive academic achievement.

\newpage

\hypertarget{methods}{%
\section{Methods}\label{methods}}

\hypertarget{sampling}{%
\subsection{Sampling}\label{sampling}}

I obtained the data from the China Education Panel Survey (CEPS) (NSRC, n.d.). It is a nationally representative, school-based survey featuring junior high school students. The survey administrator is the National Survey Research Center at Renmin University of China. The survey team adopted multi-stage stratified probability proportional to size sampling and administered a paper-and-pencil questionnaire to each participant. The baseline survey took place in the 2013-2014 academic year with a sample size of about twenty thousand students in the seventh and ninth grades. These students were nested in 438 classrooms of 112 schools in 28 urban districts or rural counties. The follow-up survey in the 2014-2015 academic year lost track of the ninth grade cohort and only surveyed 9449 of the seventh grade cohort. I further restricted the sample to students staying with both parents at the original domicile at the baseline survey. To deal with missing values, I adopted list-wise deletion for dependent, independent, and control variables. The final sample size amounted to 4088 .

\hypertarget{measurement}{%
\subsection{Measurement}\label{measurement}}

I explored two categories of educational performance: cognitive and academic abilities. One key dependent variable, the student's cognitive skill, was measured by a fifteen-minute standardized test. The test covered three types of reasoning: verbal, visuospatial, and numerical. The CEPS research team calculated the raw score based on the Item Response Theory and then standardized the score with a mean of zero and standard deviation of one.

Another dependent variable concerned students' academic performance. This was measured by the total scores of mid-term exams in Chinese, mathematics, and English as provided by each school. Throughout junior high school, students study these subjects, which are included in the entrance exam to senior high school. Within each cohort in one particular school, teachers adopt the same syllabus and administer the same exams during a given assessment period. Therefore, test scores for core subjects are reliable indicators of learning performance in the same school and cohort. I aligned different grading scales (full mark of 100, 120, and 150 for one subject) to the hundred-point scale. After the conversion, I checked for abnormal values outside the 0-100 range. 31 observations were outliers, accounting for less than one percent of the sample. I trimed the values of subject exam score with minimum of zero and maximum of one hundred. In this way, I can limit the impact of outliers while retaining these observations in the sample. After trimming, I aggregated the subject exam scores to obtain a total score of a 300-point scale.

The key independent variable, the parental migration status, was identified by two items in the parent survey. In both the baseline and the follow-up surveys, parents specified the family members living in the same household at the time. I constructed four treatment dummy variables measuring migration arrangements: any parents absent, only father absent, only mother absent, and both parents absent. These arrangements are mutually exclusive. The control group consisted of parents staying with their children throughout at both the baseline and the follow-up surveys. I then removed the observations that reported parental divorce or death. These type of parental absence are irrelevant to migration.

\hypertarget{data-analysis}{%
\subsection{Data analysis}\label{data-analysis}}

I adopted \texttt{R} for all the analyses, the \texttt{fixest} package for econometric modeling (Berge et al., 2022), and the \texttt{modelsummary} package for presenting the results (Arel-Bundock, 2022). I built a two-way fixed effect model in a difference-in-differences (DID) design, following Bai and colleagues' (2018; 2020) approach. The model served to analyze the educational outcome of children newly left behind by parents vis-a-vis that of children staying together with parents. I first specified an unrestricted and unadjusted model: \[\Delta score_{i,j,k} = \alpha + \beta \cdot migr_{i} + \gamma \cdot C_{j} + \lambda \cdot C_{k} + \theta \cdot score_{i,j;1} + \varepsilon_{i,j,k}\] For student \(i\) in class \(j\) in baseline and class \(k\) in endline, \(\Delta score_{i,j}\) is the change in test score between baseline and follow-up surveys, \(migr_{i}\) is the treatment dummy variable, \(C_{j}\) and \(C_{k}\) are respectively the classroom fixed-effects at baseline and follow-up, and \(score_{i,j,k;1}\) is the test score at baseline. Classroom fixed-effect can account for differences due to time-invariant factors at the classroom level and above, as students nested in the same class are also in the same school and county. About one-sixth of the students were reassigned into new classroom at the end of baseline academic year. This necessitates adding the wave-two classroom fixed-effect.

The model is unrestricted because it does not restrict on the coefficient associated with the baseline scores. The model is unadjusted as it does not adjust for additional covariates. Theoretically, it is unnecessary to include covariates that vary over group but remain constant over time; They would cancel out in the two-way fixed effect model (Huntington-Klein, n.d.). The standard errors were clustered at the classroom level.

In addition, I present an unrestricted and adjusted DID model: \[\Delta score_{i,j,k} = \alpha + \beta \cdot migr_{i} + \gamma \cdot C_{j} + \lambda \cdot C_{k} + \theta \cdot score_{i,j,k;1} + \zeta \cdot X_{i} + \varepsilon_{i,j,k}\] where \(X_{i}\) is the vector of covariates capturing the characteristics of children, their parents, and their households. The control variables were measured in the baseline survey. This version of model lifts the restriction that covariates from the baseline survey would be associated with a coefficient that equals one.

\newpage

\hypertarget{results}{%
\section{Results}\label{results}}

\hypertarget{descriptive-statistics}{%
\subsection{Descriptive Statistics}\label{descriptive-statistics}}

The table below summarizes the changes in migration status in the sample. All 4088 children were living together with parents in one household in baseline, but about fifteen percent saw at least one parent migrated during the two waves of survey. In these new migrant households, only-mother migration and only-father migration each accounted for two out of five cases. The remaining one-fifth was both-parents migration.

\begin{longtable}[]{@{}llrr@{}}
\caption{Parental migration status at follow-up survey.}\tabularnewline
\toprule
& & N & \% \\
\midrule
\endfirsthead
\toprule
& & N & \% \\
\midrule
\endhead
Parental Migration & No & 3452 & 84.4 \\
& Yes & 636 & 15.6 \\
Parental Migration Type & Both Parents At Home & 3452 & 84.4 \\
& Both Parents Migrated & 131 & 3.2 \\
& Only Father Migrated & 251 & 6.1 \\
& Only Mother Migrated & 254 & 6.2 \\
\bottomrule
\end{longtable}

The tables below provide the whole-sample summary statistics. In addition, the violin plots in the appendix visualize the distribution of outcome variables for the treatment group and the control group.

\begin{longtable}[]{@{}lrrrr@{}}
\caption{The independent variables}\tabularnewline
\toprule
& Mean & SD & Min & Max \\
\midrule
\endfirsthead
\toprule
& Mean & SD & Min & Max \\
\midrule
\endhead
Cognitive score difference & 0.27 & 0.80 & -2.88 & 4.03 \\
Cognitive score at wave 1 & 0.15 & 0.85 & -2.03 & 2.33 \\
Cognitive score at wave 2 & 0.42 & 0.78 & -3.14 & 2.06 \\
Academic exam score difference & -8.82 & 33.21 & -207.50 & 166.00 \\
Academic exam score at wave 1 & 213.11 & 49.46 & 19.33 & 297.50 \\
Academic exam score at wave 2 & 204.29 & 55.41 & 0.00 & 293.33 \\
\bottomrule
\end{longtable}

\begin{longtable}[]{@{}llrr@{}}
\caption{The control variables}\tabularnewline
\toprule
& & N & \% \\
\midrule
\endfirsthead
\toprule
& & N & \% \\
\midrule
\endhead
migr & 0 & 3452 & 84.4 \\
& 1 & 636 & 15.6 \\
child.is.girl & 0 & 2002 & 49.0 \\
& 1 & 2086 & 51.0 \\
child.has.rural.hukou & 0 & 2221 & 54.3 \\
& 1 & 1867 & 45.7 \\
child.in.boarding.school & 0 & 2906 & 71.1 \\
& 1 & 1182 & 28.9 \\
child.is.ethnic.minority & 0 & 3800 & 93.0 \\
& 1 & 288 & 7.0 \\
child.had.skipped.grade & 0 & 4042 & 98.9 \\
& 1 & 46 & 1.1 \\
child.had.repeated.grade & 0 & 3700 & 90.5 \\
& 1 & 388 & 9.5 \\
child.went.to.preschool & 0 & 639 & 15.6 \\
& 1 & 3449 & 84.4 \\
parent.has.white.collar.job & 0 & 3238 & 79.2 \\
& 1 & 850 & 20.8 \\
home.has.internet & 0 & 1316 & 32.2 \\
& 1 & 2772 & 67.8 \\
child.health.self.rated & Very poor & 21 & 0.5 \\
& Not very good & 113 & 2.8 \\
& Moderate & 807 & 19.7 \\
& Good & 1383 & 33.8 \\
& Very good & 1764 & 43.2 \\
child.educational.aspiration & Junior high school & 140 & 3.4 \\
& Senior high school & 418 & 10.2 \\
& Associate college & 616 & 15.1 \\
& University & 1205 & 29.5 \\
& Postgraduate & 1709 & 41.8 \\
parent.educational.level & Primary school and below & 248 & 6.1 \\
& Junior high school & 1580 & 38.6 \\
& Senior high school & 1254 & 30.7 \\
& Associate college & 385 & 9.4 \\
& University & 543 & 13.3 \\
& Postgraduate & 78 & 1.9 \\
parent.expectation.on.child & Junior high school & 36 & 0.9 \\
& Senior high school & 257 & 6.3 \\
& Associate college & 426 & 10.4 \\
& University & 1574 & 38.5 \\
& Postgraduate & 1795 & 43.9 \\
home.extracurricular.book & Very few & 395 & 9.7 \\
& Not many & 414 & 10.1 \\
& Some & 1312 & 32.1 \\
& Quite a few & 1111 & 27.2 \\
& A great number & 856 & 20.9 \\
home.economic.resource & Very poor & 115 & 2.8 \\
& Somewhat poor & 598 & 14.6 \\
& Moderate & 3124 & 76.4 \\
& Somewhat rich & 239 & 5.8 \\
& Very rich & 12 & 0.3 \\
\bottomrule
\end{longtable}

\hypertarget{regression-analysis}{%
\subsection{Regression Analysis}\label{regression-analysis}}

The unrestricted DID models support my hypothesis one -- newly left-behind adolescents are more likely to perform worse in cognitive and academic abilities than their non left-behind counterparts. If a household sees any parent migrated, the child's cognitive test score and total exam grade would decrease relative to that of children whose parents stay with them, holding everything else constant. Parental migration is linked to is -0.09 SD in cognitive test score and -0.09 points in academic exam grades. Both relationships are slight in strength.

The results only partially support the hypothesis two -- mother migration has a more negative effect than father migration. This holds true for academic abilities. Children from only mother migrate households are associated with -0.04 points relative to that of children whose parents stay along. The performances of children with both parents migrated or only father migrated do not show statistically significant difference from the non left-behind children. Regarding cognitive abilities, the reverse is true. Children with only mother migrated do not show statistically significant difference from the non left-behind children. Children with only father absent are associated with (-0.14 SD), and both parents absent (-0.10 SD) compared to the non left-behind children.

\begin{longtable}[]{@{}
  >{\raggedright\arraybackslash}p{(\columnwidth - 8\tabcolsep) * \real{0.1398}}
  >{\centering\arraybackslash}p{(\columnwidth - 8\tabcolsep) * \real{0.2043}}
  >{\centering\arraybackslash}p{(\columnwidth - 8\tabcolsep) * \real{0.2151}}
  >{\centering\arraybackslash}p{(\columnwidth - 8\tabcolsep) * \real{0.2151}}
  >{\centering\arraybackslash}p{(\columnwidth - 8\tabcolsep) * \real{0.2258}}@{}}
\caption{Parental migration's effect on children's cognitive abilities}\tabularnewline
\toprule
\begin{minipage}[b]{\linewidth}\raggedright
\end{minipage} & \begin{minipage}[b]{\linewidth}\centering
Any Parent Absent
\end{minipage} & \begin{minipage}[b]{\linewidth}\centering
Only Mother Absent
\end{minipage} & \begin{minipage}[b]{\linewidth}\centering
Only Father Absent
\end{minipage} & \begin{minipage}[b]{\linewidth}\centering
Both Parents Absent
\end{minipage} \\
\midrule
\endfirsthead
\toprule
\begin{minipage}[b]{\linewidth}\raggedright
\end{minipage} & \begin{minipage}[b]{\linewidth}\centering
Any Parent Absent
\end{minipage} & \begin{minipage}[b]{\linewidth}\centering
Only Mother Absent
\end{minipage} & \begin{minipage}[b]{\linewidth}\centering
Only Father Absent
\end{minipage} & \begin{minipage}[b]{\linewidth}\centering
Both Parents Absent
\end{minipage} \\
\midrule
\endhead
Explanator & -0.09** & -0.04 & -0.14** & -0.10 \\
& (0.03) & (0.05) & (0.04) & (0.06) \\
W1 score & -0.62*** & -0.63*** & -0.62*** & -0.62*** \\
& (0.02) & (0.02) & (0.02) & (0.02) \\
Num.Obs. & 4088 & 3706 & 3703 & 3583 \\
R2 & 0.494 & 0.502 & 0.505 & 0.499 \\
R2 Adj. & 0.424 & 0.426 & 0.430 & 0.419 \\
R2 Within & 0.370 & 0.378 & 0.378 & 0.374 \\
FE: clsids & X & X & X & X \\
FE: w2clsids & X & X & X & X \\
\bottomrule
\end{longtable}

\textbf{Note:}
\^{}\^{} † p \textless{} 0.1, * p \textless{} 0.05, ** p \textless{} 0.01, *** p \textless{} 0.001

\begin{longtable}[]{@{}
  >{\raggedright\arraybackslash}p{(\columnwidth - 8\tabcolsep) * \real{0.1398}}
  >{\centering\arraybackslash}p{(\columnwidth - 8\tabcolsep) * \real{0.2043}}
  >{\centering\arraybackslash}p{(\columnwidth - 8\tabcolsep) * \real{0.2151}}
  >{\centering\arraybackslash}p{(\columnwidth - 8\tabcolsep) * \real{0.2151}}
  >{\centering\arraybackslash}p{(\columnwidth - 8\tabcolsep) * \real{0.2258}}@{}}
\caption{Parental migration's effect on children's academic abilities}\tabularnewline
\toprule
\begin{minipage}[b]{\linewidth}\raggedright
\end{minipage} & \begin{minipage}[b]{\linewidth}\centering
Any Parent Absent
\end{minipage} & \begin{minipage}[b]{\linewidth}\centering
Only Mother Absent
\end{minipage} & \begin{minipage}[b]{\linewidth}\centering
Only Father Absent
\end{minipage} & \begin{minipage}[b]{\linewidth}\centering
Both Parents Absent
\end{minipage} \\
\midrule
\endfirsthead
\toprule
\begin{minipage}[b]{\linewidth}\raggedright
\end{minipage} & \begin{minipage}[b]{\linewidth}\centering
Any Parent Absent
\end{minipage} & \begin{minipage}[b]{\linewidth}\centering
Only Mother Absent
\end{minipage} & \begin{minipage}[b]{\linewidth}\centering
Only Father Absent
\end{minipage} & \begin{minipage}[b]{\linewidth}\centering
Both Parents Absent
\end{minipage} \\
\midrule
\endhead
Explanator & -2.67* & -5.85* & -0.25 & -1.13 \\
& (1.26) & (2.36) & (1.66) & (2.28) \\
W1 score & 0.04 & 0.04† & 0.04† & 0.04 \\
& (0.02) & (0.02) & (0.02) & (0.02) \\
Num.Obs. & 4088 & 3706 & 3703 & 3583 \\
R2 & 0.562 & 0.577 & 0.566 & 0.577 \\
R2 Adj. & 0.501 & 0.513 & 0.500 & 0.510 \\
R2 Within & 0.005 & 0.009 & 0.004 & 0.004 \\
FE: clsids & X & X & X & X \\
FE: w2clsids & X & X & X & X \\
\bottomrule
\end{longtable}

\textbf{Note:}
\^{}\^{} † p \textless{} 0.1, * p \textless{} 0.05, ** p \textless{} 0.01, *** p \textless{} 0.001

The adjusted DID models produce similar results compared to the unadjusted models, with only one exception. That is, for the cognitive skills, the both-parent-absent coefficient saw a change of statistical significance level from five percent level to ten percent level.

Some predictors are strong in both models of cognitive and academic abilities. These are children's age and educational aspiration as well as parents' expectation on children. In terms of age, the older students' scores drop relatively more than younger students'. One year older in age is associated with -0.07 SD for standardized cognitive test and -1.23 points for academic exams. Parent's expectation for children completing university and children's own aspiration for finishing university are associated with higher scores. The effect sizes are 0.01 SD and 0.07 SD for standardized cognitive test, respectively; and 0.29 points and 1.82 points for academic exams, respectively.

Several predictors are strong in the model of academic grades but not in that of cognitive scores. These are children's gender and self-reported health, and the years of parental education. Girls outperform boys by 4.49 points in academic exams, but do not show advantage in cognitive test. Students who consider themselves healthier are associated with higher scores in academic exams. Parents with more education are associated with their children's higher scores in academic exams. If the years of parental education of the best-educated among the two increase by one, it would be linked to a 0.88 points increase in the child's academic exam grade. These three predictors are not significant for cognitive outcomes.

Preschool attendance is associated with higher cognitive outcomes 0.07 SD, but its effect for academic outcomes is not significant.

\begin{longtable}[]{@{}
  >{\raggedright\arraybackslash}p{(\columnwidth - 8\tabcolsep) * \real{0.2661}}
  >{\centering\arraybackslash}p{(\columnwidth - 8\tabcolsep) * \real{0.1743}}
  >{\centering\arraybackslash}p{(\columnwidth - 8\tabcolsep) * \real{0.1835}}
  >{\centering\arraybackslash}p{(\columnwidth - 8\tabcolsep) * \real{0.1835}}
  >{\centering\arraybackslash}p{(\columnwidth - 8\tabcolsep) * \real{0.1927}}@{}}
\caption{Parental migration's effect on children's academic abilities}\tabularnewline
\toprule
\begin{minipage}[b]{\linewidth}\raggedright
\end{minipage} & \begin{minipage}[b]{\linewidth}\centering
Any Parent Absent
\end{minipage} & \begin{minipage}[b]{\linewidth}\centering
Only Mother Absent
\end{minipage} & \begin{minipage}[b]{\linewidth}\centering
Only Father Absent
\end{minipage} & \begin{minipage}[b]{\linewidth}\centering
Both Parents Absent
\end{minipage} \\
\midrule
\endfirsthead
\toprule
\begin{minipage}[b]{\linewidth}\raggedright
\end{minipage} & \begin{minipage}[b]{\linewidth}\centering
Any Parent Absent
\end{minipage} & \begin{minipage}[b]{\linewidth}\centering
Only Mother Absent
\end{minipage} & \begin{minipage}[b]{\linewidth}\centering
Only Father Absent
\end{minipage} & \begin{minipage}[b]{\linewidth}\centering
Both Parents Absent
\end{minipage} \\
\midrule
\endhead
Explanator & -0.09** & -0.04 & -0.14*** & -0.08 \\
& (0.03) & (0.05) & (0.04) & (0.06) \\
W1 score & -0.67*** & -0.67*** & -0.67*** & -0.67*** \\
& (0.02) & (0.02) & (0.02) & (0.02) \\
child.age & -0.07*** & -0.06** & -0.07*** & -0.08*** \\
& (0.02) & (0.02) & (0.02) & (0.02) \\
child.is.girl & -0.01 & 0.00 & 0.00 & -0.01 \\
& (0.02) & (0.02) & (0.02) & (0.02) \\
child.health.self.rated & -0.02 & -0.02 & -0.02 & -0.02 \\
& (0.01) & (0.01) & (0.01) & (0.01) \\
child.has.rural.hukou & -0.01 & 0.00 & -0.01 & 0.00 \\
& (0.02) & (0.03) & (0.03) & (0.03) \\
child.number.of.sibling & -0.03 & -0.02 & -0.02 & -0.02 \\
& (0.02) & (0.02) & (0.02) & (0.02) \\
child.went.to.preschool & 0.07* & 0.06† & 0.05† & 0.07* \\
& (0.03) & (0.03) & (0.03) & (0.03) \\
child.had.skipped.grade & -0.17 & -0.11 & -0.20 & -0.10 \\
& (0.12) & (0.12) & (0.12) & (0.12) \\
child.had.repeated.grade & -0.08* & -0.11** & -0.07† & -0.06 \\
& (0.04) & (0.04) & (0.04) & (0.04) \\
child.in.boarding.school & 0.04 & 0.02 & 0.02 & 0.01 \\
& (0.04) & (0.04) & (0.04) & (0.04) \\
child.educational.aspiration & 0.07*** & 0.07*** & 0.07*** & 0.08*** \\
& (0.01) & (0.01) & (0.01) & (0.01) \\
child.is.ethnic.minority & 0.04 & 0.04 & 0.02 & 0.02 \\
& (0.06) & (0.06) & (0.06) & (0.06) \\
parent.expectation.on.child & 0.08*** & 0.09*** & 0.09*** & 0.09*** \\
& (0.02) & (0.02) & (0.02) & (0.02) \\
parent.has.white.collar.job & 0.01 & -0.01 & 0.01 & 0.00 \\
& (0.03) & (0.03) & (0.03) & (0.03) \\
parent.educational.level & 0.01 & 0.01 & 0.01 & 0.01 \\
& (0.01) & (0.01) & (0.01) & (0.01) \\
home.economic.resource & -0.02 & -0.01 & 0.00 & 0.00 \\
& (0.02) & (0.02) & (0.02) & (0.02) \\
home.extracurricular.book & 0.01 & 0.00 & 0.01 & 0.00 \\
& (0.01) & (0.01) & (0.01) & (0.01) \\
home.has.internet & -0.01 & -0.01 & -0.02 & -0.02 \\
& (0.03) & (0.03) & (0.03) & (0.03) \\
Num.Obs. & 4088 & 3706 & 3703 & 3583 \\
R2 & 0.525 & 0.533 & 0.537 & 0.530 \\
R2 Adj. & 0.458 & 0.459 & 0.464 & 0.452 \\
R2 Within & 0.409 & 0.416 & 0.418 & 0.413 \\
FE: clsids & X & X & X & X \\
FE: w2clsids & X & X & X & X \\
\bottomrule
\end{longtable}

\textbf{Note:}
\^{}\^{} † p \textless{} 0.1, * p \textless{} 0.05, ** p \textless{} 0.01, *** p \textless{} 0.001

\begin{longtable}[]{@{}
  >{\raggedright\arraybackslash}p{(\columnwidth - 8\tabcolsep) * \real{0.2661}}
  >{\centering\arraybackslash}p{(\columnwidth - 8\tabcolsep) * \real{0.1743}}
  >{\centering\arraybackslash}p{(\columnwidth - 8\tabcolsep) * \real{0.1835}}
  >{\centering\arraybackslash}p{(\columnwidth - 8\tabcolsep) * \real{0.1835}}
  >{\centering\arraybackslash}p{(\columnwidth - 8\tabcolsep) * \real{0.1927}}@{}}
\caption{Parental migration's effect on children's academic abilities}\tabularnewline
\toprule
\begin{minipage}[b]{\linewidth}\raggedright
\end{minipage} & \begin{minipage}[b]{\linewidth}\centering
Any Parent Absent
\end{minipage} & \begin{minipage}[b]{\linewidth}\centering
Only Mother Absent
\end{minipage} & \begin{minipage}[b]{\linewidth}\centering
Only Father Absent
\end{minipage} & \begin{minipage}[b]{\linewidth}\centering
Both Parents Absent
\end{minipage} \\
\midrule
\endfirsthead
\toprule
\begin{minipage}[b]{\linewidth}\raggedright
\end{minipage} & \begin{minipage}[b]{\linewidth}\centering
Any Parent Absent
\end{minipage} & \begin{minipage}[b]{\linewidth}\centering
Only Mother Absent
\end{minipage} & \begin{minipage}[b]{\linewidth}\centering
Only Father Absent
\end{minipage} & \begin{minipage}[b]{\linewidth}\centering
Both Parents Absent
\end{minipage} \\
\midrule
\endhead
Explanator & -2.65* & -5.87** & -0.20 & -0.92 \\
& (1.25) & (2.24) & (1.68) & (2.35) \\
W1 score & -0.01 & -0.01 & 0.00 & -0.01 \\
& (0.02) & (0.03) & (0.02) & (0.02) \\
child.age & -1.23† & -1.04 & -1.21† & -0.96 \\
& (0.65) & (0.68) & (0.70) & (0.70) \\
child.is.girl & 4.49*** & 4.78*** & 4.54*** & 4.43*** \\
& (0.92) & (0.95) & (0.92) & (0.97) \\
child.health.self.rated & 1.05* & 1.12* & 0.92* & 0.85† \\
& (0.43) & (0.45) & (0.47) & (0.47) \\
child.has.rural.hukou & 1.08 & 1.50 & 0.80 & 0.98 \\
& (1.02) & (1.06) & (1.05) & (1.07) \\
child.number.of.sibling & 1.03† & 0.87 & 1.21* & 0.66 \\
& (0.58) & (0.59) & (0.60) & (0.62) \\
child.went.to.preschool & 1.65 & 2.22† & 0.85 & 1.97 \\
& (1.20) & (1.24) & (1.28) & (1.27) \\
child.had.skipped.grade & -0.73 & 0.23 & -0.98 & 1.01 \\
& (4.82) & (4.82) & (5.41) & (5.07) \\
child.had.repeated.grade & -1.81 & -1.22 & -1.55 & -0.20 \\
& (1.85) & (1.98) & (1.89) & (1.81) \\
child.in.boarding.school & -1.14 & -0.19 & -0.72 & -1.19 \\
& (1.65) & (1.65) & (1.66) & (1.81) \\
child.educational.aspiration & 1.82*** & 2.08*** & 1.90*** & 1.91*** \\
& (0.52) & (0.55) & (0.51) & (0.53) \\
child.is.ethnic.minority & 0.48 & -0.24 & -0.33 & -1.14 \\
& (2.15) & (2.29) & (2.38) & (2.56) \\
parent.expectation.on.child & 0.88† & 0.85 & 0.86 & 0.96† \\
& (0.53) & (0.55) & (0.54) & (0.55) \\
parent.has.white.collar.job & 0.29 & 0.03 & 0.59 & 0.42 \\
& (1.19) & (1.23) & (1.21) & (1.25) \\
parent.educational.level & 1.02† & 1.14* & 1.14* & 1.15* \\
& (0.53) & (0.53) & (0.54) & (0.56) \\
home.economic.resource & -0.74 & -0.18 & -0.66 & -0.24 \\
& (0.81) & (0.88) & (0.87) & (0.87) \\
home.extracurricular.book & -0.78† & -1.18* & -0.86† & -0.85† \\
& (0.45) & (0.46) & (0.45) & (0.46) \\
home.has.internet & -1.41 & -1.65 & -1.33 & -1.37 \\
& (1.13) & (1.14) & (1.18) & (1.17) \\
Num.Obs. & 4088 & 3706 & 3703 & 3583 \\
R2 & 0.573 & 0.590 & 0.578 & 0.588 \\
R2 Adj. & 0.512 & 0.525 & 0.512 & 0.520 \\
R2 Within & 0.032 & 0.039 & 0.032 & 0.030 \\
FE: clsids & X & X & X & X \\
FE: w2clsids & X & X & X & X \\
\bottomrule
\end{longtable}

\textbf{Note:}
\^{}\^{} † p \textless{} 0.1, * p \textless{} 0.05, ** p \textless{} 0.01, *** p \textless{} 0.001

\hypertarget{matching}{%
\subsection{Matching}\label{matching}}

Parental migration is not a randomly assigned treatment. One technique to tackle this selection bias is matching. This method works by pairing a unit in the treatment group with one or several units in the control group based on observable covariates. Popular choices include coarsened exact matching (CEM) (Waldman, 2022) and nearest-neighbor propensity score matching (PSM) (H. Xu \& Xie, 2015). As Greifer (2022) introduced, CEM is superior in achieving balance as it optimizes the entire joint distribution of covariates. The downside is that it leaves many units without an exact match and discards these unmatched units. The nearest-neighbor matching examines all the treated units and selects the closest control unit to be paired. The eligibility as the ``closest'' can be determined with the differences in propensity score. Propensity score denotes the probability of receiving treatment based on the specified covariates. Nearest-neighbor PSM does not try to optimize any condition; each pairing ignores how other pairing will or have occurred (Greifer, 2022). King and Nielsen (2019) argued against this matching method as it risks more ``imbalance, model dependence, and bias'' for highly balanced data.

Using the \texttt{MatchIt} package (Ho et al., 2022), I performed matching with all the covariates of the adjusted DID model. I first attempted CEM, but it left most units unmatched and resulted in a tiny sample. I then made do with the PSM. I combined exact matching (based on the student's county of residence) and nearest neighbor PSM (based on the other covariates) following the configuration of Bai et al. (2018). Exact matching based on county of residence preserves the original geographic distribution. The propensity score was estimated using a logistic regression of the treatment on the covariates. The matching ratio was set at one treatment unit with up to three control units. The matching produced absolute standardized mean differences below 0.1 for all but a few covariates, suggesting an improved overall balance.

Alike are the results from DID models with matching (DIDM) and those without. In the models of cognitive abilities, I obtained negative, statistically significant coefficients on the treatment variables in the any parent migrated household and only father migrated households. In the models of academic abilities, the treatment variables' coefficients are negative and statistically significant for the any parent migrated households and only mother migrated households. The significance level of any parent migration drops to ten percent. In all models, the magnitudes of the statistically significant coefficients are similar. I present the unadjusted DIDM models below and the adjusted DIDM models in the appendix.

\begin{longtable}[]{@{}
  >{\raggedright\arraybackslash}p{(\columnwidth - 8\tabcolsep) * \real{0.1398}}
  >{\centering\arraybackslash}p{(\columnwidth - 8\tabcolsep) * \real{0.2043}}
  >{\centering\arraybackslash}p{(\columnwidth - 8\tabcolsep) * \real{0.2151}}
  >{\centering\arraybackslash}p{(\columnwidth - 8\tabcolsep) * \real{0.2151}}
  >{\centering\arraybackslash}p{(\columnwidth - 8\tabcolsep) * \real{0.2258}}@{}}
\caption{Parental migration's effect on children's cognitive abilities, estimated with matching}\tabularnewline
\toprule
\begin{minipage}[b]{\linewidth}\raggedright
\end{minipage} & \begin{minipage}[b]{\linewidth}\centering
Any Parent Absent
\end{minipage} & \begin{minipage}[b]{\linewidth}\centering
Only Mother Absent
\end{minipage} & \begin{minipage}[b]{\linewidth}\centering
Only Father Absent
\end{minipage} & \begin{minipage}[b]{\linewidth}\centering
Both Parents Absent
\end{minipage} \\
\midrule
\endfirsthead
\toprule
\begin{minipage}[b]{\linewidth}\raggedright
\end{minipage} & \begin{minipage}[b]{\linewidth}\centering
Any Parent Absent
\end{minipage} & \begin{minipage}[b]{\linewidth}\centering
Only Mother Absent
\end{minipage} & \begin{minipage}[b]{\linewidth}\centering
Only Father Absent
\end{minipage} & \begin{minipage}[b]{\linewidth}\centering
Both Parents Absent
\end{minipage} \\
\midrule
\endhead
Explanator & -0.09** & -0.05 & -0.13* & -0.05 \\
& (0.03) & (0.06) & (0.06) & (0.11) \\
W1 score & -0.61*** & -0.65*** & -0.60*** & -0.63*** \\
& (0.03) & (0.04) & (0.04) & (0.06) \\
Num.Obs. & 1958 & 913 & 897 & 475 \\
R2 & 0.528 & 0.627 & 0.638 & 0.646 \\
R2 Adj. & 0.375 & 0.313 & 0.320 & -0.312 \\
R2 Within & 0.363 & 0.407 & 0.372 & 0.351 \\
FE: clsids & X & X & X & X \\
FE: w2clsids & X & X & X & X \\
\bottomrule
\end{longtable}

\textbf{Note:}
\^{}\^{} † p \textless{} 0.1, * p \textless{} 0.05, ** p \textless{} 0.01, *** p \textless{} 0.001

\textbf{Note:}
\^{}\^{} 3:1 nearest neighbor propensity score matching with replacement; exact matching by county

\begin{longtable}[]{@{}
  >{\raggedright\arraybackslash}p{(\columnwidth - 8\tabcolsep) * \real{0.1398}}
  >{\centering\arraybackslash}p{(\columnwidth - 8\tabcolsep) * \real{0.2043}}
  >{\centering\arraybackslash}p{(\columnwidth - 8\tabcolsep) * \real{0.2151}}
  >{\centering\arraybackslash}p{(\columnwidth - 8\tabcolsep) * \real{0.2151}}
  >{\centering\arraybackslash}p{(\columnwidth - 8\tabcolsep) * \real{0.2258}}@{}}
\caption{Parental migration's effect on children's academic abilities, estimated with matching}\tabularnewline
\toprule
\begin{minipage}[b]{\linewidth}\raggedright
\end{minipage} & \begin{minipage}[b]{\linewidth}\centering
Any Parent Absent
\end{minipage} & \begin{minipage}[b]{\linewidth}\centering
Only Mother Absent
\end{minipage} & \begin{minipage}[b]{\linewidth}\centering
Only Father Absent
\end{minipage} & \begin{minipage}[b]{\linewidth}\centering
Both Parents Absent
\end{minipage} \\
\midrule
\endfirsthead
\toprule
\begin{minipage}[b]{\linewidth}\raggedright
\end{minipage} & \begin{minipage}[b]{\linewidth}\centering
Any Parent Absent
\end{minipage} & \begin{minipage}[b]{\linewidth}\centering
Only Mother Absent
\end{minipage} & \begin{minipage}[b]{\linewidth}\centering
Only Father Absent
\end{minipage} & \begin{minipage}[b]{\linewidth}\centering
Both Parents Absent
\end{minipage} \\
\midrule
\endhead
Explanator & -2.65† & -5.34* & -2.42 & -2.45 \\
& (1.47) & (2.67) & (2.56) & (3.96) \\
W1 score & 0.03 & 0.08† & 0.05 & 0.03 \\
& (0.03) & (0.04) & (0.04) & (0.06) \\
Num.Obs. & 1958 & 913 & 897 & 475 \\
R2 & 0.590 & 0.682 & 0.637 & 0.706 \\
R2 Adj. & 0.457 & 0.413 & 0.319 & -0.087 \\
R2 Within & 0.006 & 0.026 & 0.008 & 0.005 \\
FE: clsids & X & X & X & X \\
FE: w2clsids & X & X & X & X \\
\bottomrule
\end{longtable}

\textbf{Note:}
\^{}\^{} † p \textless{} 0.1, * p \textless{} 0.05, ** p \textless{} 0.01, *** p \textless{} 0.001

\textbf{Note:}
\^{}\^{} 3:1 nearest neighbor propensity score matching with replacement; exact matching by county

\newpage

\hypertarget{discussion}{%
\section{Discussion}\label{discussion}}

This thesis proposed a model to assess causality between parents' migration and their stay-behind children's educational performance in the short run. I built econometric models in the difference-in-differences design to analyze data from the 2013-2014 and 2014-2015 rounds of China Education Panel Survey. Results suggest that parental migration is, in the short run, negatively associated with children's educational performance as measured by academic exams and standardized cognitive tests. I further examined whether the effect varies across different arrangements of parental migration. Results indicate that father-only migration is negatively associated with children's cognitive abilities, while mother-only migration is negatively associated with children's academic abilities. In all scenarios, the educational performance of newly left-behind children only decreased slightly compared to that of non-left-behind peers.

My research has limitations in sampling due to data constraint. Regarding sample size, the treatment group for the both-parents migration is small, lowering the statistical power. Regarding identification, I could only identify parental migration through such a condition: parental absence from a household without parental divorce or death. This is because the CEPS questionnaire asked if parents were living in the household, but did not ask why parents were absent. It is possible that some parents were jailed or hospitalized. These types of absence would be wrongly attributed to migration. Besides, I could not control for the migration history of parents, which are not included in the CEPS data. Possibly, some parents out-migrated at an earlier date, returned before the baseline survey, and migrated away again during the survey period. Their children might better cope with parental absence during the survey period, yet they would be mistaken as the newly left-behind. These limitations may bias the estimation.

Limitations also arise from unaccounted endogeneity and exogeneity. As Chen et al. (2009) stated in their research on left-behind children, endogenous factors may create selection bias. When migration opportunities arise, some parents may believe that leaving children behind would hamper children's education, thereby giving up the opportunity. Some other parents may believe that their children's schooling would not suffer from parental absence, thereby embarking on migration. Parental beliefs of this sort are unavailable in CEPS data. Exogenous factors may create omitted variable bias. Suppose that a natural hazard hit many counties of a province in the sample. This shock can simultaneously increase parents' out-migration and reduce stay-behind students' grades (Chen et al., 2009). I could not account for such shocks as CEPS data does not include detailed geographic information.

From my research, we can find new directions for scholarly inquiry. First, when the CEPS team releases the next round of data, researchers can capitalize a the three-wave panel dataset to build multi-period DID models. This would facilitate examination of the parallel trend assumption. Second, researchers can try novel methods. Machine learning, for example, can estimate the heterogeneous effects across subgroups (R. Liu, 2021). Synthetic control can evaluate the aggregate outcome of ``large-scale but infrequent interventions'' at municipal or provincial levels (Abadie, 2021), such as the policy experimentations that liberalize the hukou system in certain cities (X. Wang, 2020). Third, researchers can go beyond outbound migration to study return migration of parents (Liang \& Li, 2021). Only a few publications, such as that of Z. Liu et al. (2018), have explored return migrants and their stay-behind children in the Chinese context.

Let us consider the policy implications for China. The small reduction of newly left behind children's schooling performance by no means suggest that policies can remain as they are. The slight reduction is only the short-term consequence. Longer duration of parental migration may hamper educational performance to a greater extent. Besides, educational performance is only part of human development outcomes (Antia et al., 2020). Other outcomes like mental health and non-cognitive skills are also subjected to parental absence's influence. Policymakers should reduce institutional barriers at destination cities, opening more doors for parents to migrate together with their children. Meanwhile, schools should dedicate more advisory and counseling support to children who stay behind.

\newpage

\hypertarget{appendix}{%
\section{Appendix}\label{appendix}}

\begin{figure}
\centering
\includegraphics{MScThesisText_files/figure-latex/unnamed-chunk-10-1.pdf}
\caption{\label{fig:unnamed-chunk-10}Changes in standardized cognitive test score for newly left-behind students and non-left-behind students}
\end{figure}

\newpage

\begin{figure}
\centering
\includegraphics{MScThesisText_files/figure-latex/unnamed-chunk-11-1.pdf}
\caption{\label{fig:unnamed-chunk-11}Changes in academic exam score for newly left-behind students and non-left-behind students}
\end{figure}

\newpage

\begin{figure}
\centering
\includegraphics{MScThesisText_files/figure-latex/unnamed-chunk-12-1.pdf}
\caption{\label{fig:unnamed-chunk-12}Changes in standardized cognitive test score by household migration arrangement}
\end{figure}

\newpage

\begin{figure}
\centering
\includegraphics{MScThesisText_files/figure-latex/unnamed-chunk-13-1.pdf}
\caption{\label{fig:unnamed-chunk-13}Changes in academic exam score by household migration type}
\end{figure}

\newpage

\begin{longtable}[]{@{}
  >{\raggedright\arraybackslash}p{(\columnwidth - 8\tabcolsep) * \real{0.2661}}
  >{\centering\arraybackslash}p{(\columnwidth - 8\tabcolsep) * \real{0.1743}}
  >{\centering\arraybackslash}p{(\columnwidth - 8\tabcolsep) * \real{0.1835}}
  >{\centering\arraybackslash}p{(\columnwidth - 8\tabcolsep) * \real{0.1835}}
  >{\centering\arraybackslash}p{(\columnwidth - 8\tabcolsep) * \real{0.1927}}@{}}
\caption{Parental migration's effect on children's academic abilities, estimated with matching}\tabularnewline
\toprule
\begin{minipage}[b]{\linewidth}\raggedright
\end{minipage} & \begin{minipage}[b]{\linewidth}\centering
Any Parent Absent
\end{minipage} & \begin{minipage}[b]{\linewidth}\centering
Only Mother Absent
\end{minipage} & \begin{minipage}[b]{\linewidth}\centering
Only Father Absent
\end{minipage} & \begin{minipage}[b]{\linewidth}\centering
Both Parents Absent
\end{minipage} \\
\midrule
\endfirsthead
\toprule
\begin{minipage}[b]{\linewidth}\raggedright
\end{minipage} & \begin{minipage}[b]{\linewidth}\centering
Any Parent Absent
\end{minipage} & \begin{minipage}[b]{\linewidth}\centering
Only Mother Absent
\end{minipage} & \begin{minipage}[b]{\linewidth}\centering
Only Father Absent
\end{minipage} & \begin{minipage}[b]{\linewidth}\centering
Both Parents Absent
\end{minipage} \\
\midrule
\endhead
Explanator & -0.09** & -0.04 & -0.15* & -0.08 \\
& (0.03) & (0.06) & (0.06) & (0.11) \\
W1 score & -0.67*** & -0.67*** & -0.67*** & -0.69*** \\
& (0.03) & (0.04) & (0.04) & (0.06) \\
child.age & -0.08* & -0.02 & -0.03 & -0.15 \\
& (0.03) & (0.05) & (0.04) & (0.10) \\
child.is.girl & 0.02 & 0.01 & 0.05 & -0.06 \\
& (0.03) & (0.05) & (0.06) & (0.10) \\
child.health.self.rated & 0.00 & -0.03 & -0.02 & -0.07 \\
& (0.02) & (0.03) & (0.03) & (0.06) \\
child.has.rural.hukou & 0.01 & -0.07 & 0.06 & 0.04 \\
& (0.04) & (0.07) & (0.07) & (0.12) \\
child.number.of.sibling & -0.07* & -0.09* & -0.05 & -0.10 \\
& (0.03) & (0.04) & (0.04) & (0.07) \\
child.went.to.preschool & 0.04 & 0.08 & 0.03 & 0.30* \\
& (0.04) & (0.08) & (0.06) & (0.12) \\
child.had.skipped.grade & -0.08 & -0.13 & -0.37† & 0.34 \\
& (0.16) & (0.24) & (0.19) & (0.45) \\
child.had.repeated.grade & -0.08 & -0.26* & -0.02 & 0.05 \\
& (0.06) & (0.11) & (0.11) & (0.16) \\
child.in.boarding.school & 0.09 & 0.15 & -0.10 & -0.02 \\
& (0.07) & (0.12) & (0.11) & (0.20) \\
child.educational.aspiration & 0.07*** & 0.03 & 0.10** & 0.10† \\
& (0.02) & (0.03) & (0.03) & (0.05) \\
child.is.ethnic.minority & -0.01 & 0.01 & -0.20 & 0.39 \\
& (0.09) & (0.10) & (0.23) & (0.44) \\
parent.expectation.on.child & 0.07** & 0.08† & 0.06† & 0.01 \\
& (0.03) & (0.05) & (0.03) & (0.06) \\
parent.has.white.collar.job & 0.02 & -0.10† & 0.03 & -0.15 \\
& (0.05) & (0.05) & (0.08) & (0.16) \\
parent.educational.level & 0.01 & 0.03 & 0.00 & -0.08 \\
& (0.02) & (0.03) & (0.03) & (0.06) \\
home.economic.resource & -0.05 & 0.00 & 0.02 & 0.05 \\
& (0.03) & (0.05) & (0.05) & (0.08) \\
home.extracurricular.book & 0.01 & 0.01 & 0.02 & 0.01 \\
& (0.02) & (0.03) & (0.03) & (0.05) \\
home.has.internet & -0.05 & -0.07 & -0.06 & 0.05 \\
& (0.04) & (0.08) & (0.07) & (0.11) \\
Num.Obs. & 1958 & 913 & 897 & 475 \\
R2 & 0.562 & 0.653 & 0.672 & 0.691 \\
R2 Adj. & 0.413 & 0.337 & 0.361 & -0.318 \\
R2 Within & 0.409 & 0.447 & 0.431 & 0.435 \\
FE: clsids & X & X & X & X \\
FE: w2clsids & X & X & X & X \\
\bottomrule
\end{longtable}

\textbf{Note:}
\^{}\^{} † p \textless{} 0.1, * p \textless{} 0.05, ** p \textless{} 0.01, *** p \textless{} 0.001

\textbf{Note:}
\^{}\^{} 3:1 nearest neighbor propensity score matching with replacement; exact matching by county

\newpage

\begin{longtable}[]{@{}
  >{\raggedright\arraybackslash}p{(\columnwidth - 8\tabcolsep) * \real{0.2661}}
  >{\centering\arraybackslash}p{(\columnwidth - 8\tabcolsep) * \real{0.1743}}
  >{\centering\arraybackslash}p{(\columnwidth - 8\tabcolsep) * \real{0.1835}}
  >{\centering\arraybackslash}p{(\columnwidth - 8\tabcolsep) * \real{0.1835}}
  >{\centering\arraybackslash}p{(\columnwidth - 8\tabcolsep) * \real{0.1927}}@{}}
\caption{Parental migration's effect on children's academic abilities, estimated with matching}\tabularnewline
\toprule
\begin{minipage}[b]{\linewidth}\raggedright
\end{minipage} & \begin{minipage}[b]{\linewidth}\centering
Any Parent Absent
\end{minipage} & \begin{minipage}[b]{\linewidth}\centering
Only Mother Absent
\end{minipage} & \begin{minipage}[b]{\linewidth}\centering
Only Father Absent
\end{minipage} & \begin{minipage}[b]{\linewidth}\centering
Both Parents Absent
\end{minipage} \\
\midrule
\endfirsthead
\toprule
\begin{minipage}[b]{\linewidth}\raggedright
\end{minipage} & \begin{minipage}[b]{\linewidth}\centering
Any Parent Absent
\end{minipage} & \begin{minipage}[b]{\linewidth}\centering
Only Mother Absent
\end{minipage} & \begin{minipage}[b]{\linewidth}\centering
Only Father Absent
\end{minipage} & \begin{minipage}[b]{\linewidth}\centering
Both Parents Absent
\end{minipage} \\
\midrule
\endhead
Explanator & -2.91† & -5.90* & -2.68 & -2.94 \\
& (1.48) & (2.64) & (2.56) & (4.21) \\
W1 score & -0.01 & 0.01 & 0.00 & -0.03 \\
& (0.03) & (0.04) & (0.05) & (0.06) \\
child.age & -0.61 & 1.45 & 2.59 & -2.15 \\
& (1.21) & (2.00) & (1.99) & (3.48) \\
child.is.girl & 4.51** & 4.40† & 6.58** & 3.76 \\
& (1.43) & (2.57) & (2.41) & (4.21) \\
child.health.self.rated & 0.71 & 2.30 & 1.13 & 2.38 \\
& (0.69) & (1.42) & (1.15) & (2.27) \\
child.has.rural.hukou & 0.85 & -0.13 & -1.10 & -2.75 \\
& (1.70) & (3.10) & (2.97) & (4.45) \\
child.number.of.sibling & 1.20 & 1.92† & 1.00 & -2.50 \\
& (1.06) & (1.06) & (1.61) & (2.69) \\
child.went.to.preschool & 1.75 & -0.30 & -2.26 & 6.39 \\
& (1.76) & (2.47) & (2.57) & (5.13) \\
child.had.skipped.grade & 2.29 & -9.76 & 3.12 & 32.64 \\
& (6.62) & (11.90) & (12.03) & (29.40) \\
child.had.repeated.grade & -6.07† & -13.55† & -13.33** & -0.09 \\
& (3.18) & (7.61) & (4.86) & (5.92) \\
child.in.boarding.school & -0.64 & -0.65 & -3.42 & -3.10 \\
& (3.53) & (5.38) & (4.25) & (8.17) \\
child.educational.aspiration & 1.48† & 1.63 & 0.37 & 0.44 \\
& (0.86) & (1.06) & (1.20) & (1.92) \\
child.is.ethnic.minority & 3.81 & 4.84 & 2.79 & -2.54 \\
& (4.31) & (4.68) & (6.53) & (7.64) \\
parent.expectation.on.child & 0.19 & 1.71 & 1.41 & 3.97 \\
& (0.83) & (1.49) & (1.35) & (2.49) \\
parent.has.white.collar.job & -0.85 & 0.54 & 2.13 & -1.42 \\
& (1.82) & (2.56) & (2.87) & (6.14) \\
parent.educational.level & 1.30 & 2.30** & 1.91 & -2.63 \\
& (0.97) & (0.85) & (1.41) & (2.54) \\
home.economic.resource & -0.23 & -1.90 & -0.51 & -0.54 \\
& (1.23) & (2.24) & (1.81) & (4.05) \\
home.extracurricular.book & -0.66 & -0.66 & -1.59 & 0.08 \\
& (0.71) & (1.24) & (1.01) & (1.93) \\
home.has.internet & -2.89 & -1.61 & 0.08 & 3.67 \\
& (1.81) & (2.87) & (3.26) & (4.98) \\
Num.Obs. & 1958 & 913 & 897 & 475 \\
R2 & 0.603 & 0.702 & 0.660 & 0.731 \\
R2 Adj. & 0.468 & 0.431 & 0.337 & -0.150 \\
R2 Within & 0.037 & 0.088 & 0.070 & 0.087 \\
FE: clsids & X & X & X & X \\
FE: w2clsids & X & X & X & X \\
\bottomrule
\end{longtable}

\textbf{Note:}
\^{}\^{} † p \textless{} 0.1, * p \textless{} 0.05, ** p \textless{} 0.01, *** p \textless{} 0.001

\textbf{Note:}
\^{}\^{} 3:1 nearest neighbor propensity score matching with replacement; exact matching by county

\newpage

\hypertarget{references}{%
\section{References}\label{references}}

\hypertarget{refs}{}
\begin{CSLReferences}{1}{0}
\leavevmode\vadjust pre{\hypertarget{ref-abadie2021}{}}%
Abadie, A. (2021). Using Synthetic Controls: Feasibility, Data Requirements, and Methodological Aspects. \emph{Journal of Economic Literature}, \emph{59}(2), 391--425. \url{https://doi.org/10.1257/jel.20191450}

\leavevmode\vadjust pre{\hypertarget{ref-akezhuoli2022}{}}%
Akezhuoli, H., Lu, J., Zhao, G., Xu, J., Wang, M., Wang, F., Li, L., \& Zhou, X. (2022). Mother's and father's migrating in china: Differing relations to mental health and risk behaviors among left-behind children. \emph{Frontiers in Public Health}, \emph{10}. \url{https://www.frontiersin.org/article/10.3389/fpubh.2022.894741}

\leavevmode\vadjust pre{\hypertarget{ref-antia2020}{}}%
Antia, K., Boucsein, J., Deckert, A., Dambach, P., Račaitė, J., Šurkienė, G., Jaenisch, T., Horstick, O., \& Winkler, V. (2020). Effects of International Labour Migration on the Mental Health and Well-Being of Left-Behind Children: A Systematic Literature Review. \emph{International Journal of Environmental Research and Public Health}, \emph{17}(12), 4335. \url{https://doi.org/10.3390/ijerph17124335}

\leavevmode\vadjust pre{\hypertarget{ref-arel-bundock2022}{}}%
Arel-Bundock, V. (2022). \emph{Modelsummary: Summary tables and plots for statistical models and data: Beautiful, customizable, and publication-ready}. \url{https://vincentarelbundock.github.io/modelsummary/}

\leavevmode\vadjust pre{\hypertarget{ref-bai2020}{}}%
Bai, Y., Neubauer, M., Ru, T., Shi, Y., Kenny, K., \& Rozelle, S. (2020). Impact of Second-Parent Migration on Student Academic Performance in Northwest China and its Implications. \emph{The Journal of Development Studies}, \emph{56}(8), 1523--1540. \url{https://doi.org/10.1080/00220388.2019.1690136}

\leavevmode\vadjust pre{\hypertarget{ref-bai2018}{}}%
Bai, Y., Zhang, L., Liu, C., Shi, Y., Mo, D., \& Rozelle, S. (2018). Effect of parental migration on the academic performance of left behind children in north western china. \emph{The Journal of Development Studies}, \emph{54}(7), 1154--1170. \url{https://doi.org/10.1080/00220388.2017.1333108}

\leavevmode\vadjust pre{\hypertarget{ref-berge2022}{}}%
Berge, L., Krantz, S., \& McDermott, G. (2022). \emph{Fixest: Fast fixed-effects estimations}. \url{https://CRAN.R-project.org/package=fixest}

\leavevmode\vadjust pre{\hypertarget{ref-black2021}{}}%
Black, M. M., Behrman, J. R., Daelmans, B., Prado, E. L., Richter, L., Tomlinson, M., Trude, A. C. B., Wertlieb, D., Wuermli, A. J., \& Yoshikawa, H. (2021). The principles of Nurturing Care promote human capital and mitigate adversities from preconception through adolescence. \emph{BMJ Global Health}, \emph{6}(4), e004436. \url{https://doi.org/10.1136/bmjgh-2020-004436}

\leavevmode\vadjust pre{\hypertarget{ref-botev2019}{}}%
Botev, J., Égert, B., Smidova, Z., \& Turner, D. (2019). \emph{A new macroeconomic measure of human capital with strong empirical links to productivity}. \url{https://doi.org/10.1787/d12d7305-en}

\leavevmode\vadjust pre{\hypertarget{ref-botezat2020}{}}%
Botezat, A., \& Pfeiffer, F. (2020). The impact of parental labour migration on left{-}behind children's educational and psychosocial outcomes: Evidence from Romania. \emph{Population, Space and Place}, \emph{26}(2). \url{https://doi.org/10.1002/psp.2277}

\leavevmode\vadjust pre{\hypertarget{ref-bourdieu2002}{}}%
Bourdieu, P. (2002). \emph{The Forms of Capital} (N. W. Biggart, Ed.; pp. 280--291). Blackwell Publishers Ltd. \url{https://doi.org/10.1002/9780470755679.ch15}

\leavevmode\vadjust pre{\hypertarget{ref-debrauw2019}{}}%
Brauw, A. de. (2019). Migration Out of Rural Areas and Implications for Rural Livelihoods. \emph{Annual Review of Resource Economics}, \emph{11}(1), 461--481. \url{https://doi.org/10.1146/annurev-resource-100518-093906}

\leavevmode\vadjust pre{\hypertarget{ref-debrauw2017}{}}%
Brauw, A. de, \& Giles, J. (2017). Migrant Opportunity and the Educational Attainment of Youth in Rural China. \emph{Journal of Human Resources}, \emph{52}(1), 272--311. \url{https://doi.org/10.3368/jhr.52.1.0813-5900R}

\leavevmode\vadjust pre{\hypertarget{ref-chan2017}{}}%
Chan, J., \& Selden, M. (2017). The labour politics of china{'}s rural migrant workers. \emph{Globalizations}, \emph{14}(2), 259--271. \url{https://doi.org/10.1080/14747731.2016.1200263}

\leavevmode\vadjust pre{\hypertarget{ref-chang2019}{}}%
Chang, F., Jiang, Y., Loyalka, P., Chu, J., Shi, Y., Osborn, A., \& Rozelle, S. (2019). Parental migration, educational achievement, and mental health of junior high school students in rural China. \emph{China Economic Review}, \emph{54}, 337--349. \url{https://doi.org/10.1016/j.chieco.2019.01.007}

\leavevmode\vadjust pre{\hypertarget{ref-chen2009}{}}%
Chen, X., Huang, Q., Rozelle, S., Shi, Y., \& Zhang, L. (2009). Effect of Migration on Children's Educational Performance in Rural China. \emph{Comparative Economic Studies}, \emph{51}(3), 323--343. \url{https://doi.org/10.1057/ces.2008.44}

\leavevmode\vadjust pre{\hypertarget{ref-chen2019}{}}%
Chen, X., Li, D., Liu, J., Fu, R., \& Liu, S. (2019). Father migration and mother migration: Different implications for social, school, and psychological adjustment of left-behind children in rural china. \emph{Journal of Contemporary China}, \emph{28}(120), 849--863. \url{https://doi.org/10.1080/10670564.2019.1594100}

\leavevmode\vadjust pre{\hypertarget{ref-cunningham2021}{}}%
Cunningham, S. (2021). \emph{Causal inference: The mixtape}. Yale University Press. \url{https://mixtape.scunning.com/}

\leavevmode\vadjust pre{\hypertarget{ref-dewaard2018}{}}%
DeWaard, J., Nobles, J., \& Donato, K. M. (2018). Migration and parental absence: A comparative assessment of transnational families in Latin America. \emph{Population, Space and Place}, \emph{24}(7), e2166. \url{https://doi.org/10.1002/psp.2166}

\leavevmode\vadjust pre{\hypertarget{ref-duckett2020}{}}%
Duckett, J. (2020). Neoliberalism, Authoritarian Politics and Social Policy in China. \emph{Development and Change}, \emph{51}(2), 523--539. \url{https://doi.org/10.1111/dech.12568}

\leavevmode\vadjust pre{\hypertarget{ref-goldin2016}{}}%
Goldin, C. (2016). \emph{Human Capital} (C. Diebolt \& M. Haupert, Eds.; pp. 55--86). Springer. \url{https://doi.org/10.1007/978-3-642-40406-1_23}

\leavevmode\vadjust pre{\hypertarget{ref-greifer2022}{}}%
Greifer, N. (2022). \emph{Matching methods}. \url{https://kosukeimai.github.io/MatchIt/articles/matching-methods.html}

\leavevmode\vadjust pre{\hypertarget{ref-gu2021}{}}%
Gu, X. (2021). {`}Save the children!{'}: Governing left-behind children through family in China{'}s Great Migration. \emph{Current Sociology}, 0011392120985874. \url{https://doi.org/10.1177/0011392120985874}

\leavevmode\vadjust pre{\hypertarget{ref-ho2022}{}}%
Ho, D., Imai, K., King, G., Stuart, E., Whitworth, A., \& Greifer, N. (2022). \emph{MatchIt: Nonparametric preprocessing for parametric causal inference}. \url{https://CRAN.R-project.org/package=MatchIt}

\leavevmode\vadjust pre{\hypertarget{ref-hong2019}{}}%
Hong, Y., \& Fuller, C. (2019). Alone and {``}left behind{''}: A case study of {``}left-behind children{''} in rural china. \emph{Cogent Education}, \emph{6}(1), 1654236. \url{https://doi.org/10.1080/2331186X.2019.1654236}

\leavevmode\vadjust pre{\hypertarget{ref-huntington-klein}{}}%
Huntington-Klein, N. (n.d.). \emph{The effect: An introduction to research design and causality \textbar{} the effect}. \url{https://theeffectbook.net/index.html}

\leavevmode\vadjust pre{\hypertarget{ref-jin2020}{}}%
Jin, X., Chen, W., Sun, I. Y., \& Liu, L. (2020). Physical health, school performance and delinquency: A comparative study of left-behind and non-left-behind children in rural China. \emph{Child Abuse \& Neglect}, \emph{109}, 104707. \url{https://doi.org/10.1016/j.chiabu.2020.104707}

\leavevmode\vadjust pre{\hypertarget{ref-king2019}{}}%
King, G., \& Nielsen, R. (2019). Why Propensity Scores Should Not Be Used for Matching. \emph{Political Analysis}, \emph{27}(4), 435--454. \url{https://doi.org/10.1017/pan.2019.11}

\leavevmode\vadjust pre{\hypertarget{ref-liang2021}{}}%
Liang, Z., \& Li, W. (2021). Migration and children in China: a review and future research agenda. \emph{Chinese Sociological Review}, \emph{53}(3), 312--331. \url{https://doi.org/10.1080/21620555.2021.1908823}

\leavevmode\vadjust pre{\hypertarget{ref-liu2021a}{}}%
Liu, R. (2021). Leveraging machine learning methods to estimate heterogeneous effects: Father absence in china as an example. \emph{Chinese Sociological Review}, \emph{0}(0), 1--29. \url{https://doi.org/10.1080/21620555.2021.1948828}

\leavevmode\vadjust pre{\hypertarget{ref-liu2021}{}}%
Liu, Y., Deng, Z., \& Katz, I. (2021). Transmission of Educational Outcomes Across Three Generations: Evidence From Migrant Workers{'} Children in China. \emph{Applied Research in Quality of Life}. \url{https://doi.org/10.1007/s11482-021-09990-y}

\leavevmode\vadjust pre{\hypertarget{ref-liu2018}{}}%
Liu, Z., Yu, L., \& Zheng, X. (2018). No longer left-behind: The impact of return migrant parents on children's performance. \emph{China Economic Review}, \emph{49}, 184--196. \url{https://doi.org/10.1016/j.chieco.2017.06.004}

\leavevmode\vadjust pre{\hypertarget{ref-murphy2022}{}}%
Murphy, R. (2022). What does {`}left behind{'} mean to children living in migratory regions in rural China? \emph{Geoforum}, \emph{129}, 181--190. \url{https://doi.org/10.1016/j.geoforum.2022.01.012}

\leavevmode\vadjust pre{\hypertarget{ref-nbs2022}{}}%
NBS. (2022). \emph{2021 migrant workers monitor survey report {[}in Chinese{]}}. \url{http://www.stats.gov.cn/tjsj/zxfb/202204/t20220429_1830126.html}

\leavevmode\vadjust pre{\hypertarget{ref-nbs2017}{}}%
NBS, UNICEF, \& UNFPA. (2017). \emph{Population Status of Children in China in 2015: Facts and Figures}. \url{https://www.unicef.cn/media/9901/file/Population\%20Status\%20of\%20Children\%20in\%20China\%20in\%202015\%20Facts\%20and\%20Figures.pdf}

\leavevmode\vadjust pre{\hypertarget{ref-nguyen2016}{}}%
Nguyen, C. V. (2016). Does parental migration really benefit left-behind children? Comparative evidence from Ethiopia, India, Peru and Vietnam. \emph{Social Science \& Medicine}, \emph{153}, 230--239. \url{https://doi.org/10.1016/j.socscimed.2016.02.021}

\leavevmode\vadjust pre{\hypertarget{ref-nsrc}{}}%
NSRC. (n.d.). \emph{Overview-china education panel survey}. \url{http://ceps.ruc.edu.cn/English/Overview/Overview.htm}

\leavevmode\vadjust pre{\hypertarget{ref-popova2018}{}}%
Popova, A. (2018). Risk factors for the safety of the children from transnational families children left behind. \emph{Anthropological Researches and Studies}, \emph{1}(8), 25--35. \url{https://doi.org/10.26758/8.1.3}

\leavevmode\vadjust pre{\hypertarget{ref-shu2021}{}}%
Shu, B. (2021). Parental Migration, Parental Emotional Support, and Adolescent Children{'}s Life Satisfaction in Rural China: The Roles of Parent and Child Gender. \emph{Journal of Family Issues}, \emph{42}(8), 1663--1705. \url{https://doi.org/10.1177/0192513X20946345}

\leavevmode\vadjust pre{\hypertarget{ref-vanhook2020}{}}%
Van Hook, J., \& Glick, J. E. (2020). Spanning Borders, Cultures, and Generations: A~Decade of Research on Immigrant Families. \emph{Journal of Marriage and Family}, \emph{82}(1), 224--243. \url{https://doi.org/10.1111/jomf.12621}

\leavevmode\vadjust pre{\hypertarget{ref-waldman2022}{}}%
Waldman, K. E. (2022). Transnational Social Stratification? Legal Status of Immigrant Parents and the Educational Achievements of Mexican Children. \emph{International Migration Review}, 01979183221084329. \url{https://doi.org/10.1177/01979183221084329}

\leavevmode\vadjust pre{\hypertarget{ref-wang2019}{}}%
Wang, L., Zheng, Y., Li, G., Li, Y., Fang, Z., Abbey, C., \& Rozelle, S. (2019). Academic achievement and mental health of left-behind children in rural china: A causal study on parental migration. \emph{China Agricultural Economic Review}, \emph{11}(4), 569--582. \url{https://doi.org/10.1108/CAER-09-2018-0194}

\leavevmode\vadjust pre{\hypertarget{ref-wang2020}{}}%
Wang, X. (2020). Permits, Points, and Permanent Household Registration: Recalibrating Hukou Policy under {``}Top-Level Design{''}. \emph{Journal of Current Chinese Affairs}, \emph{49}(3), 269--290. \url{https://doi.org/10.1177/1868102619894739}

\leavevmode\vadjust pre{\hypertarget{ref-xu2015}{}}%
Xu, H., \& Xie, Y. (2015). The Causal Effects of Rural-to-Urban Migration on Children{'}s Well-being in China. \emph{European Sociological Review}, \emph{31}(4), 502--519. \url{https://doi.org/10.1093/esr/jcv009}

\leavevmode\vadjust pre{\hypertarget{ref-xu2019}{}}%
Xu, Y., Xu, D., Simpkins, S., \& Warschauer, M. (2019). Does It Matter Which Parent is Absent? Labor Migration, Parenting, and Adolescent Development in China. \emph{Journal of Child and Family Studies}, \emph{28}(6), 1635--1649. \url{https://doi.org/10.1007/s10826-019-01382-z}

\leavevmode\vadjust pre{\hypertarget{ref-yang2008}{}}%
Yang, D. (2008). International Migration, Remittances and Household Investment: Evidence from Philippine Migrants{'} Exchange Rate Shocks. \emph{The Economic Journal}, \emph{118}(528), 591--630. \url{https://doi.org/10.1111/j.1468-0297.2008.02134.x}

\leavevmode\vadjust pre{\hypertarget{ref-yang2022}{}}%
Yang, F., Wang, Z., Liu, J., \& Tang, S. (2022). The Effect of Social Interaction on Rural Children{'}s Self-identity: an Empirical Study Based on Survey Data from Jintang County, China. \emph{Child Indicators Research}, \emph{15}(3), 839--861. \url{https://doi.org/10.1007/s12187-021-09887-0}

\leavevmode\vadjust pre{\hypertarget{ref-zhang2018}{}}%
Zhang, C. (2018). Governing neoliberal authoritarian citizenship: Theorizing hukou and the changing mobility regime in china. \emph{Citizenship Studies}, \emph{22}(8), 855--881. \url{https://doi.org/10.1080/13621025.2018.1531824}

\end{CSLReferences}


\end{document}
